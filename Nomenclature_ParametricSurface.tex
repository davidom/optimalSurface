I need a parametric surface so that I can maintain a valid topology.
Projection is not required since the topology gives me uv bounds. The
mesh generation is going on in parametric space. There is an
optimization function for creating the mesh in parametric space. Parametric
space is also great for optimization since you can write the topology
and optimization function constraints in uv space. Mapping a
discretization to a surface in general involved the determination of
``what'' an edge means on the surface---whether it's a geodesic line, or
what... If you can do this that's great but the development of a map
is not the purpose of this work so, for ease of presentation, a
parametric surface is used.

There's plenty of precedent for surface mesh generation in parameterized space. 

The real justification for using a parameterization is that the formulation of constraints for representation deficit is straightforward. The search space for the local optimization problems is straightforward and intuitive to define--as well as efficient to search. However, if the surface mesh (in ${\mathbb R}^3$) were mapped to the surface...[SOMEHOW REMEMBER WHERE YOU WERE GOING AND FINISH THIS THOUGHT]

Consider an orientable, parameterized surface, $\vec{S}\left(u,v\right)
: {\mathbb R}^2 \rightarrow {\mathbb R}^3$.
