Consider a discretization, $D$, defined on $S$, comprised of $n_t$
points, $P_i \in \left\{p_1,...,p_{n_t} \right\}$, and a
non-overlapping, non-degenerate, consistently oriented, triangular
connectivity $T$. Each triangle in $T$, $t_i$, is defined by points
$\left(p_j, p_k, p_m\right)$, and edges $\left(e_n, e_o,
e_p\right)$---where the edges are defined (for $t_i$) as $e_n
\left(p_j, p_k\right)$, $e_o \left(p_k, p_m\right)$, $e_p \left(p_m,
p_j\right)$. For the purposes of defining an edge in this work, the
ordering of the nodes does not matter. However, the ordering is used to
maintain consistent orientation during topological operations and to
define constraints in the developed optimization strategy.

Each point, $p_i$, in $D$ is defined at a $\left(u,v\right)_i$
coordinate, and edge is a straight line in both parametric space and in
${\mathbb R}^3$. This is not the case for a discretization which is
mapped onto a surface. In the absence of a parameterization the mapping
of an edge in ${\mathbb R}^3$ to a curve (possibly a straight line, but
not guaranteed) on a surface is an ambiguous task which is outside
the scope of this work. Additionally, the development of topological
constraints for optimization, which will be detailed later, would not be
as straightforward as is possible when using a parameterized, planar
space.[MAYBE MORE ON THIS AND MAYBE MOVE THIS TO ANOTHER SECTION OR
REORDER TOPICS]

\subsubsection{Toplogy}
Valid topology will be maintained at all times during the mesh
generation process. This will be accomplished by ordering each the
connectivity for each triangle in $T$ such that it has a positive
normal, it is straightforward to maintain valid (non-overlapping)
topology during optimization. This is done by not allowing a
topological change, such as an edge swap or nodal movement, that
creates invalid topology.

