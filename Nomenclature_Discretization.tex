Consider a discretization, $D$, defined on $S$, comprised of $n_p$
points, $P \in \left\{p_1,...,p_{n_p} \right\}$, and a non-overlapping,
non-degenerate, consistently oriented, conformal, triangular
connectivity $T \in \left\{t_1,...,t_{n_t} \right\}$. Each triangle in
$T$, $t_i$, is defined by points $\left(p_j, p_k, p_m\right)$, and edges
$\left(e_n, e_o, e_p\right)$---where the edges are defined. For the purposes of
defining an edge in this work, the ordering of the nodes does not matter.
However, the ordering of the triangle connectivity (sometimes referred to as
winding) is used to maintain consistent orientation during topological
operations and to define constraints in the developed optimization strategy.

Each point, $p_i$, in $D$ is defined at a $\left(u,v\right)_i$
coordinate, and an edge is a straight line in both parametric space and in
${\mathbb R}^3$. This is not the case for a discretization, $D:{\mathbb
R}^3$, which is mapped onto a surface, $S:{\mathbb R}^3$. In the absence
of a parameterization the mapping of an edge in ${\mathbb R}^3$ to a
curve (possibly a straight line, but not guaranteed) on a surface is an
ambiguous task which is outside the scope of this work. Additionally,
the development of topological constraints for optimization, which will
be detailed later, would not be as straightforward as is possible when
using a parameterized, planar space.
