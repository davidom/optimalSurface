Let us develop the definition of a local optimization problem using the
concept of representation deficit as an objective function. First
consider all of the variables: the number of points ($n_p$), the
optimized topology/connectivity ($T_o$), and the position/location of
the points ($\vec{N_i}$).

\subsection{Fixed $n_p$, Fixed $T_o$, Dyanamic $\vec{N_i}$}
To start, let us define the problem as simply as possible: a fixed
number of points, a fixed topology, and the points have a variable
position.  The local optimization problem is defined as follows: [For
each node, $n_i$, determine the surface area of the topologically
adjacent triangles, $A_0$. Calculate the objective function for each of
the four candidate new positions in the pattern, $\left(O_{up},
O_{down}, O_{left}, O_{right}\right)$. Determine the minimum value from
the objective functions calculations, $O_{min}$. Compare $O_{min}$ to
the representation deficit change limit. Move if appropriate]
[DEFINITION OF OPTIMIZATION PROBLEMS]
[HOW TO FIT IN THE EDGE FLIPPING AS AN OPTIMIZATION PROBLEM? I CAN
JUSTIFY IT VIA HAND-WAVING, BUT NEED SOME WAY TO EXPLAIN IT BETTER]

\subsubsection{Race Condition}
In practice it was observed that nodes will often move toward each other
when there is a ``local minimum'' nearby. Here this phenomenon is called
a {\it race-condition}. However, once the two nodes get sufficiently
near to each other one will inevitably get to the local minimum first
and the other node will be ``pushed'' the away.

\subsection{Fixed $n_p$, Dynamic $T_o$, Dyanamic $\vec{N_i}$}
\subsubsection{Edge Flipping}
Another phenomenon that was noticed was that during the smoothing the
nodes would often be repositioned near edges. This usually meant that
the nearby minimum was on the edge or on the other topological side of
the edge. Without some sort of intervention, this would lead to nearly
degenerate geometry being formed. In order to prevent geometry which
could cause numerical robustness issues, when a node neared an edge the
edge was flipped. This allowed the node to move unobstructed by the
newly-reconnected edge.

The criteria for determining if a node is ``too close'' to an edge is
based on the pattern. If the minimum value for the objective function is
on or across an edge, then the edge is flipped and the process is
repeated with the new topology.

\subsection{Dynamic $n_p$, Dynamic $T_o$, Fixed $\vec{N_i}$}
This formulation is not well-posed in that without some sort of limit on
the number of nodes to add to the discretization {\it the} optimal
triangulation has infinite points. An infinitely refined surface would
have no discretization error. However, there is a limiting term that
ensures that infinite refinement will not occur: representation-deficit
limit [RENAME?]. This method of optimization takes advantage of the
optimal substructure associated with this optimization problem. Each
triangle is considered and the optimal nodal position is calculated
using the aforementioned pattern matching. However, there are some
limits on where these points can reside. If the pattern-matching leads
to a nodal position is too close to an edge then it is rejected.

\subsection{Dynamic $n_p$, Dynamic $T_o$, Dynamic $\vec{N_i}$}
This formulation combines all of the above modes by iteratively
combining representation-deficit driven refinement with nodal smoothing
and local reconnection.

%Most works seek to refine based on element quality and some sort of
%sizing or grading algorithm--which again seeks to transition from one
%size to another based on a pre-defined quality metric. This works for
%the most part because the surface grid generator is really not worried
%about being over tesselated--only creating good quality elements to be
%used downstream by a volume grid generator.
%
%This work seeks to explore how to ``best'' represent a continuous
%surface with a discrete representation. To define ``best'', some
%criteria of ``good'' has to be established. First consider two trias
%(embedded in r3, blah blah), how can we say that one is a ``better''
%representation of the underlying surface than another one? Well that is
%a question that can only be quantified by calculating how well the
%triangle represents the portion of the surface to which is assigned.
%Here we define representation deficit as determining whether a triangle
%represents a surface ``better'' than another triangle.
%
%The concept of representation deficit can also be extended to a
%collection of triangles. This is useful and will be shown later how this
%is used during fundamental mesh operations.
%
%[PARAGRAPH ABOUT OPTIMAL SUBSTRUCTURE]
%
%At a basic level, representation deficit for a surface grid is the
%difference between the surface area of a discretization and the surface
%area of a parameterized surface. For this work, the existence of a
%parameterization is important to the next point: If the surface area of
%EACH PIECE (triangle, edge) of the discretization is maximized while
%maintaining valid and consistent topology (each triangle has positive
%area in the oriented parameterization space) then the surface area
%of the discretization must approach the surface area of the
%parameterized surface. Relaxing the above restriction that the
%triangulation must be valid would not allow a discrete surface with
%infinite area and therefore must be maintained. Therefore, we can
%conclude that any operation that increases the surface area of the
%discrete surface must be ``going in the right direction.''
