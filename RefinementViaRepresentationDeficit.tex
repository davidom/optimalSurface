Most works seek to refine based on element quality and some sort of sizing or grading algorithm--which again seeks to transition from one size to another based on a pre-defined quality metric. This works for the most part because the surface grid generator is really not worried about being over tesselated--only creating good quality elements to be used downstream by a volume grid generator.

This work seeks to explore how to "best" represent a continuous surface with a discrete representation. To define "best", some criteria of "good" has to be established. First consider two trias (embedded in r3, blah blah), how can we say that one is a "better" representation than another one? Well that is a question that can only be quantified by calculating how well the triangle represents the portion of the surface to which is assigned. Here we define representation deficit as determining whether a triangle represents a surface "better" than another triangle.

The concept of representation deficit can also be extended to a collection of triangles. This is useful and will be shown later how this is used during fundamental mesh operations.

At a basic level, representation deficit for a surface grid is the difference between the surface area of a discretization and the surface area of a parameterized surface. For this work, the existence of a parameterization is important to the next point: If the surface area of EACH PIECE (triangle, edge) of the discretization is maximized while maintaining valid and consistent topology (each triangle has positive area in the oriented parameterization space) then the surface area of the discretization must approach the surface area of the parameterized surface. Relaxing the above restriction that the triangulation must be valid would not allow a discrete surface with infinite area and therefore must be maintained. Therefore, we can conclude that any operation that increases the surface area of the discrete surface must be “going in the right direction.”
