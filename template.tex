%%%%%%%%%%%%%%%%%%%%%%% file template.tex %%%%%%%%%%%%%%%%%%%%%%%%%
%
% This is a general template file for the LaTeX package SVJour3
% for Springer journals.          Springer Heidelberg 2010/09/16
%
% Copy it to a new file with a new name and use it as the basis
% for your article. Delete % signs as needed.
%
% This template includes a few options for different layouts and
% content for various journals. Please consult a previous issue of
% your journal as needed.
%
%%%%%%%%%%%%%%%%%%%%%%%%%%%%%%%%%%%%%%%%%%%%%%%%%%%%%%%%%%%%%%%%%%%
%
%
\RequirePackage{fix-cm}
%
%\documentclass{svjour3}                     % onecolumn (standard format)
%\documentclass[smallcondensed]{svjour3}     % onecolumn (ditto)
\documentclass[smallextended]{svjour3}       % onecolumn (second format)
%\documentclass[twocolumn]{svjour3}          % twocolumn
%
\smartqed  % flush right qed marks, e.g. at end of proof
%
\usepackage{graphicx}
%
% \usepackage{mathptmx}      % use Times fonts if available on your TeX system
%
% insert here the call for the packages your document requires
\usepackage{latexsym}
\usepackage{geometry}
\usepackage{amssymb}
\usepackage{amsmath}
\usepackage{algorithm}
\usepackage{algpseudocode}
\usepackage{graphicx}
%\usepackage{cleveref}
\usepackage{url}
\usepackage{subfig}
% etc.
%
% please place your own definitions here and don't use \def but
% \newcommand{}{}
%
% Insert the name of "your journal" with
\journalname{myjournal}
%
\begin{document}

\title{Representation Deficit Based Optimal Surface Mesh Generation
\thanks{The work of the second author is supported in part by NSF CAREER grant ACI-1330056 (formerly ACI-1054459)}
}
%\titlerunning{Short form of title}        % if too long for running head

\author{
  David O. McLaurin \and
  Suzanne M. Shontz
}

%\authorrunning{Short form of author list} % if too long for running head

\institute{David O. McLaurin \at
           10800 Pecan Park Blvd. \\
           Austin, TX 78750 \\
           Tel.: +1 512-331-2808 \\
           Fax: +1 512-258-5938 \\
           \email{david.mclaurin@cd-adapco.com}           %  \\
           \and
           Suzanne M. Shontz \at
              second address
}

\date{Received: date / Accepted: date}
% The correct dates will be entered by the editor

\maketitle

\begin{abstract}
Insert your abstract here. Include keywords, PACS and mathematical
subject classification numbers as needed.
\keywords{First keyword \and Second keyword \and More}
% \PACS{PACS code1 \and PACS code2 \and more}
% \subclass{MSC code1 \and MSC code2 \and more}
\end{abstract}

\section{Introduction}
Most mesh generation algorithms refine an initial discretization based on
a priori error estimates [EXAMPLE REFERENCES]. This approach,
historically and in a modern sense, has been spectacularly successful in
their efforts to produce high quality meshes for computation simulation.
Aforementioned mesh generation strategies focus on element quality ---
with the justification being that downstream applications require high
quality geometries in order to achieve a desired level of accuracy.
However, element quality should be secondary to accurately representing
the underlying physical object or phenomenon.  This work seeks to
explore the concept of optimal mesh representation instead of optimal
mesh quality.

Here a formal definition of fundamental mesh operations (triangle split,
edge split, edge collapse, and vertex smoothing) is developed with
respect to optimal representation for a continuous, parameterized
surface. The proposed algorithm is a general-use method that can be
applied to any surface regardless of its representation. This is due to
every step being developed without the use of derivatives. A result of
not using derivatives is that each step in the algorithm is robust to
large or discontinuous changes in derivatives.

This process can only be automated if some way of judging ``how well'' a
surface grid represents a surface is present. To this end, a method of
generating surface grids through constrained, local optimization is
detailed below. The above concepts are contrasted with traditional
methods for grid generation through direct comparison of representation
optimality and efficiency. Further discussion of robustness and mesh
quality is also presented.


\section{Nomenclature and Definitions}
\subsection{Parametric Surface}
In the field of surface mesh generation there exists an overwhelming
precedent for mapping physical space to parametric space and performing
the actual mesh generation in parametric space [CITE]. This strategy has
been proven to be efficient and accurate---with the assumption that the
parameterization is of reasonable quality. Here, mesh generation in
parametric space also enables the development of unambiguous
optimization constrains and objective functions. The search space for
the local optimization problem is straightforward and intuitive to
define---as well as efficient to search. Therefore, this method will be
developed for an orientable, parameterized surface,
$\vec{S}\left(u,v\right) : {\mathbb R}^2 \rightarrow {\mathbb R}^3$


\subsection{Discretization}
Consider a discretization, $D$, defined on $S$, comprised of $n_p$
points, $P \in \left\{p_1,...,p_{n_p} \right\}$, and a non-overlapping,
non-degenerate, consistently oriented, conformal, triangular
connectivity $T \in \left\{t_1,...,t_{n_t} \right\}$. Each triangle in
$T$, $t_i$, is defined by points $\left(p_j, p_k, p_m\right)$, and edges
$\left(e_n, e_o, e_p\right)$. For the purposes of defining an edge in this
work, the ordering of the nodes does not matter.  However, the ordering of the
triangle connectivity (sometimes referred to as winding) is used to maintain
consistent orientation during topological operations and to define constraints
in the developed optimization strategy.

Each point, $p_i$, in $D$ is defined at a parametric coordinate,
$\left(u,v\right)_i$, and an edge is a straight line in both parametric
space and in ${\mathbb R}^3$. This is not the case for a discretization,
$D:{\mathbb R}^3$, which is mapped onto a surface, $S:{\mathbb R}^3$. In the
absence of a parameterization the mapping of an edge in ${\mathbb R}^3$ to a
curve (possibly a straight line, but not guaranteed) on a surface is an
ambiguous task which is outside the scope of this work.  Additionally, the
development of topological constraints for optimization, which will be detailed
later, would not be as straightforward as is possible when using a
parameterized, planar space.




\section{Discretization Error, Discrete Curvature Approximation}
Discretization error can be defined for surface grids be comparing the "amount" of discrete representation to the "amount" of continuous representation.

For a line segment, or edge, a comparison should be made between the arc length of the edge and the arc length of the portion of the curve which the edge represents.

For a triangle (the only guaranteed-planar element) (get into why quads and higher elements are their own form of parametric surface and they should not be considered here because of their nebulous definition) a comparison should be made between the area of the triangle and the area of the portion of the surface which the triangle represents.

However, on a surface these two are coupled. If only the edge length of the edges is considered then it is demonstrable that the edges could have little to no representation deficit but the triangles which are defined by these edges are very poor representations of the underlying surface. If only the area of triangles is considered then a very probable outcome is the formation of degenerate triangles with zero area.


\section{Representation Deficit based Mesh Operations}
\subsection{Representation Deficit}
The concept of representation deficit is discussed in \cite{mclaurin13}
in the context of scale-independent measure of curve discretizations.
Here, the concept will be extended to two dimensions. It is important
that any criteria for refinement maintain scale independence. This is so
that the values for representation deficit can be compared between
different candidate operations. Below, representation deficit is defined
and discussed for four fundamental mesh operations: triangle split, edge
split, edge swap, and node smoothing.

\subsection{Triangle Split}
A triangle is a planar object, and is representing a possibly non-planar
portion of a surface. Any triangle, with area $A_T$, in the
discretization can have at most the same area as the portion of the
surface which it represents, $A_S$. Therefore, the representation
deficit for a triangle, $RD_T$, can be defined as $RD_T = \frac{A_S -
A_T}{A_T}$. Note that the areas are calculated in ${\mathbb R}^3$. The
difference in surface area, $\left(A_S - A_T\right)$, is normalized by
$A_T$ so that the result is scale independent. Also, this is a
representation {\it deficit} since $A_S \ge A_T$ is always true.

In order to apply the above definition of representation deficit to a
mesh generation, a replacement for $A_S$ must be determined since the
area of the surface represented by $A_S$ is not always able to be
determined --- or, most often, the area calculation is impractical.
Generally, in order to reduce the representation deficit for a triangle
the triangle is split by inserting an interior point. Any point that is
inserted into the interior of the triangle would decrease the
representation deficit --- or at worst it will remain the same. However,
the determination of where to split the triangle should be done in such
a way that the representation deficit is minimized. This strategy of
refinement, refining each triangle in such a way that the representation
deficit is locally minimized, would take advantage of the optimal
substructure the discrete topology.

\begin{figure}[h!]
  \center{\includegraphics[height=1.4in]
    {Figures/TriangleSplit.png}}
  \caption{Triangle Split}
\end{figure}

The process of determining where to split a triangle is defined here by
a locally optimizing an objective function defined for a triangle: Let
$S(u,v)$ be a parameterized surface, $D$ be be a discretization of the
surface, and $T$ be a triangle (included in $D$) defined by an ordered
tuple of nodes, $\left(n_i, n_j, n_k\right)$. These nodes are ordered
such that the triangle normal is positive. Specifically, if $\vec{V_0} =
\left\{n_j - n_i \right\}$ and $\vec{V_1} = \left\{n_k - n_i\right\}$
then the triangle normal, $N_T = \vec{V_0} \times \vec{V_1}$, is
positive. Note that a two-dimensional cross-product (or two-dimensional
curl) is a scalar quantity. Additionally, let a node on the interior of
$T$ be defined as $n_O$. The four nodes, $n_i$, $n_j$, $n_k$, and $n_O$
define three triangles, $T_i\left(n_i,n_j,n_0\right), T_j\left(n_j, n_k,
n_0\right), T_k\left(n_k, n_i, n_0\right)$. If $A(T)$ is a function
that calculates the area of a triangle in $\left(x,y,z\right)$ space,
then the optimization problem for finding the optimal position for $n_0$
in $T$ can be stated as:

\begin{eqnarray*}
\begin{array}{rcl}
\underset{n_O}{\text{minimize}} \ O(T) & = & - \frac{A\left(T_i\right) + A\left(T_j\right) + A\left(T_k\right) }{ A\left(T\right) }\\
\text{subject to} \ N_{T_i} & > & 0 \\
N_{T_j} & > & 0 \\ 
N_{T_k} & > & 0 \\
\end{array}
\end{eqnarray*}

It should be noted that this definition of representation deficit would
be difficult to derive or express for a topological entity other than a
triangle. This is due to the inherent ambiguity in the definition of not
only the surface area, but also the surface representation of (possibly)
non-planar elements, e.g. non-planar, or skew, quadrilateral. [MORE
POSSIBLY]

\subsection{Edge Split}
An edge in parametric space represents a (possible) curve on the
surface. Any edge, with length $L_E$, can have at most the same length
as the portion of the surface which it represents, $L_S$. Therefore, the
representation deficit for an edge, $RD_E$, is defined as $RD_E =
\frac{L_S - L_E}{L_E}$. Note that the lengths are calculated in
${\mathbb R}^3$. The difference in length, $\left(L_S - L_E\right)$, is
normalized by $L_E$ so that the result is scale independent. Also, this
is representation {\it deficit} since $L_S \ge L_E$ is always true.

In order to apply the above definition of representation deficit to mesh
generation, a replacement for $L_S$ must be determined since the length
along the surface represented by $L_S$ is not always able to be
determined -- or, most often, the arc-length calculation is impractical.
Generally, in order to reduce the representation deficit for an edge, the
edge is split by inserting an interior node. Any node that is inserted
into the interior of the edge would decrease the representation deficit
--- or, at worst, it will remain the same. However, the determination of
where to split the edge should be done in such a way that the
representation deficit is minimized. This strategy of refinement,
i.e., refining each edge in such a way that the representation deficit is
locally minimized, would take advantage of the optimal substructure of
the discrete topology. A method for generating a locally optimal edge
split is detailed in \cite{mclaurin12,mclaurin13}.

\begin{figure}[h!]
  \center{\includegraphics[height=1.4in]
    {Figures/EdgeSplit.jpg}}
  \caption{Edge Split}
\end{figure}

In addition, the fact that an edge split will change the surface area of
the discretization should be considered. Since the overall 
motivation of this work is to reduce the {\it area} representation deficit 
of a discretization, the representation deficit will not be defined for an
edge but for an edge-split. Therefore, for a given edge, only the
edge-split that minimizes {\it area} representation deficit is
considered. The process of determining where to split a triangle is
defined here by locally optimizing an objective function defined for an
edge-split.  Let $E\left(n_i,n_j\right)$ be an edge in $D$ that is
topologically adjacent to two triangles, $T_i\left(n_i,n_j,n_k\right)$
and $T_j\left(n_i,n_l,n_j\right)$. Additionally, let a node on the
interior of the edge be defined as $n_O$. The five nodes,
$\left\{n_i,n_j,n_k,n_l,n_O\right\}$ define four triangles, i.e.,
$T_a\left(n_i,n_O,n_k\right)$, $T_b\left(n_O,n_j, n_k\right)$,
$T_c\left(n_i,n_l,n_O\right)$, $T_d\left(n_O,n_l,n_j\right)$.  The
optimization problem for finding the optimal position for $n_O$ on $E$
is defined as:

\begin{eqnarray*}
\begin{array}{rcl}
\underset{n_O}{\text{minimize}} \ O(E) & = & - \frac{ A\left(T_a\right) + A\left(T_b\right) + A\left(T_c\right) + A\left(T_d\right) }{ A\left(T_i\right) + A\left(T_j\right) } \\
\text{subject to} \ N_{T_a} & > & 0 \\
N_{T_b} & > & 0 \\ 
N_{T_c} & > & 0 \\
N_{T_d} & > & 0. \\
\end{array}
\end{eqnarray*}

\subsection{Edge Swap}
Is it possible to have an optimal topology? If so, then it should be
reachable through repeated edge flipping. Therefore, we should be able
to write some sort of formulation for a pair of triangles and go from
there? [SUZANNE] Comment: for the sake of not doing any more coding, if
your ethics can let you come to the conclusion that this is either not
possible or not practical then this would fit into the
logic/justification in the next section.

\subsection{Nodal Movement}
In this paper, the focus is on using nodal movement in 
order to locally minimize the representation deficit in the surface mesh.  
It should be noted that, although a locally optimal mesh topology could 
also be obtained, computing such a topology would involve the solution of 
a discrete optimization problem (via integer programming) which would 
specify the relevant sequence of edge flips.  The discrete 
and continuous optimization problems could then be solved in an iterative, 
interleaving fashion.  However, such an approach would require significant 
additional computational expense.

For a node, $N_i$, the representation deficit is defined only by
comparing it to the node when located at another point in space. Here
the comparison is bound by limiting the range of comparison within the
edge-hull topologically adjacent to $N_i$. Formally, let $N_i$ be a node
in $D$ that is shared topologically by $n$ triangles, where $n$ is the
face-valence of $N_i$. The optimization for finding the optimal position
for $N_i$ is defined as:

\begin{eqnarray*}
\begin{array}{rcl}
\underset{N_O}{\text{minimize}} \ O(N) & = &
-\sum{_{j=1}^{n_t-1}A\left(T_j\right)} \\
\text{subject to} \ N_{T_1} & > & 0 \\
N_{T_2} & > & 0 \\ 
N_{T_3} & > & 0 \\
& \vdots & \\
N_{T_{n-1}} & > & 0 \\ 
N_{T_n} & > & 0.
\end{array}
\end{eqnarray*}

\begin{figure}[h!]
  \center{\includegraphics[height=1.4in]
    {Figures/NodalSmoothing.jpg}}
  \caption{Nodal Movement}
\end{figure}

%[NEED TO DISCUSS HESSIAN/METRIC BASED MESH ADAPTION IN THE NODAL
%SMOOTHING SECTION OF THE PAPER? THERE IS NO GLOBAL, STATIC FUNCTION
%AGAINST WHICH THE INTERPOLATION ERROR CAN BE MINIMIZED. THAT IS,
%WHENEVER THE MESH MOVES, THE INTERPOLATION ERROR FIELD CHANGES.]


\subsection{Boundary Refinement}
[Is there a section needed on this? It'd be straightforward to talk about
but would be pretty repetitive and I don't have it coded up yet. If I do
talk about it then due diligence says I should show some results and
there are already tons of variables to consider. This might get left out
due to space concerns.]
[This should be an argument for applying [CITE SELF] to the boundary
curves first because the interior of the surface is not what the
boundary curves are really trying to represent]

\section{Element Quality}
Element quality is not considered here, only the optimal representation
of the underlying geometry. Methods for triangular mesh improvement are
myriad---so no effort was made to contribute to that area of research.
[REWORD]


\section{Mesh Refinement Algorithm}
[NEED TO COME TO THE CONCLUSION HERE THAT GLOBAL OPTIMIZATION IS NOT
PRACTICAL] [COMMENT ON HOW ITERATIVE REFINEMENT IS USED HERE, NOT
SETTING UP SOME GLOBAL EQUATIONS]
\subsection{Local Optimization via Pattern Matching}
[Suzanne] I used a simple "cross" pattern (up down left right center)
and sized it based on the local geometry so that it could be moved
inside of the triangles/hulls many times before encountering the
boundary.


\input{MeshRefinementAlgorithmSection}

\section{Results}
In this explorative work, results will be shown for the how each
individual mesh operation (i.e., triangle split, edge split, and node 
movement) affects the representation deficit. In addition, an effective
combination of the aforementioned mesh operations was developed, and the
results will be shown. The geometry which is being discretized is the
``peaks'' function which has a well-known parametrization
\cite{peaksMatlab}. The domain was chosen so that the boundary is
sufficiently far away from the local minima/maxima present near the origin to
show that the developed scheme is efficient at reducing representation
deficit. The $x$ and $y$ values are each confined between $-5.0$ and
$5.0$ and the initial triangulation is two triangles that fill the
quadrilateral defined by the four corner nodes.

The main driver of representation deficit, $RD$, based mesh refinement
is the value for representation deficit. This is a ratio of the
``before'' area to the ``after'' area for each mesh operation. The
surface area of the peaks function in the given domain was calculated by
uniformly triangulating the surface with a very fine resolution. The
result was found to converge to $177.944$ with a $u$ and $v$ resolution
of 301 nodes. For ease of presentation, the peaks functions is shown as
a contour plot where red represents local maxima, and blue represents
local minima/maxima.

\subsection{Triangle and Edge Splitting}
Restricting the mesh refinement to only triangle splits or only edges
splits is not useful, as discussed above. Therefore, the combination of
the two was performed (as described in Fig. \ref{alg_IterativeRefinement})
to produce the following results. The following figure shows the results
from $tol_{RD}=\left(0.5,0.25,0.125\right)$ from left to right. The mesh was
refined by splitting triangles and edges where appropriate as described
above.

\begin{figure}[h!]
  \begin{center}

  \subfloat[$tol_{RD}=0.5$]{\label{fig_RefineOnlyA}\includegraphics[width=50mm]{Figures/RefineOnly05.png}}
  \subfloat[$tol_{RD}=0.25$]{\label{fig_RefineOnlyB}\includegraphics[width=50mm]{Figures/RefineOnly025.png}}
  \subfloat[$tol_{RD}=0.125$]{\label{fig_RefineOnlyC}\includegraphics[width=50mm]{Figures/RefineOnly0125.png}}

  \caption{Mesh Refinement Only}
  \label{fig_RefineOnly}
  \end{center}
\end{figure}

In Fig. \ref{fig_RefineOnlyA}, it can be seen that only three nodes were
inserted. This is due to the very high value of $tol_{RD}$ for this
example. However, nodes were inserted very near to the local
minima/maxima of the function. If $tol_{RD}$ were halved to $0.25$, Fig.
\ref{fig_RefineOnlyB} then $92$ nodes are inserted. Again, the nodes are
inserted very near to local minima/maxima and additionally near saddle
nodes. If $tol_{RD}$ is halved again to $0.125$, Fig.
\ref{fig_RefineOnlyC}, no more nodes are inserted. This is due to the
very poor quality triangles. The sliver triangles that can be seen in
Fig. \ref{fig_RefineOnlyB} and Fig. \ref{fig_RefineOnlyC} represent a
nearly planar section of the mesh and therefore is not refined further.
The major shortcoming of the developed algorithm is that element quality
is not considered. This limitation leads to poor quality meshes that
cannot be refined further.

\subsection{Node Movement Effect}
In order to show the effect of node movement, the resultant mesh shown
in Fig. \ref{fig_RefineOnlyB} is used as input, and the same value for
$tol_{RD} = 0.25$ is used with Fig. \ref{alg_NodeSmoothing}.

\begin{figure}[h!]
  \begin{center}
 
  \subfloat[$tol_{RD}=0.25$]{\label{fig_NodeSmoothingA}\includegraphics[width=60mm]{Figures/RefineOnly025.png}}
  \subfloat[$tol_{RD}=0.25$]{\label{fig_NodeSmoothingB}\includegraphics[width=60mm]{Figures/NodeSmooth025.png}}
  \caption{Effect of Node Movement on RD and Mesh Quality}
  \label{fig_NodeSmoothing}

  \end{center}
\end{figure}

Node movement was very effective at reducing representation deficit.
For this particular case, it reduced the representation deficit of the
input mesh by $23.36\%$. It should also be noted that the local
minima/maxima were accurately maintained by this procedure. It can also
be seen that moved nodes that were nearby but not in local minima/maxim
were moved to the local minima/maxima.  Additionally, since the node
movement was allowed to perform local reconnections where needed
(discussed above), the element quality was also improved as a
side-effect.

\subsection{Effective Combination}
It is easily noted that while the aforementioned mesh operations are
effective at reducing representation deficit, they do not create high
quality triangles. However, with the addition of a Delaunay local
reconnection pass in between refinement and node movement passes, the
element quality and representation deficit can be improved
significantly.

\begin{figure}[h!]
  \begin{center}
  \subfloat[$tol_{RD}=0.25$, no movement]{\label{fig_ComboA}\includegraphics[width=50mm]{Figures/RefineOnly025.png}}
  \subfloat[$tol_{RD}=0.25$, movement post]{\label{fig_ComboB}\includegraphics[width=50mm]{Figures/NodeSmooth025.png}}
  \subfloat[$tol_{RD}=0.25$, reconnection]{\label{fig_ComboC}\includegraphics[width=50mm]{Figures/Reconnect025.png}}
  \caption{Comparison of Exclusive Splitting, with addition of Node
Movement, with further addition of local reconnection}
  \label{fig_Combo}
  \end{center}
\end{figure}

\begin{figure}[h!]
  \begin{center}
  \includegraphics[width=90mm]{Figures/MeshQuality.png}
  \caption{Triangle interior angle histogram}
  \label{fig_MeshQuality}
  \end{center}
\end{figure}

In Fig. \ref{fig_ComboC}, the marked difference in mesh quality and mesh
density caused by local reconnection can be seen. The addition of the
local reconnection enabled the triangle quality to be improved every
step and therefore allowed more nodes to be inserted. In this case,
instead of $92$ nodes being inserted (Fig. \ref{fig_ComboA}), $713$
nodes were inserted. The surface area of the final mesh was $179.234$.
This is a representation deficit of $0.725\%$.  Fig.
\ref{fig_MeshQuality} shows a histogram of interior angles vs percentage
of all interior angles of the mesh results shown in Fig.
\ref{fig_Combo}. The blue line represents results from only using
triangle and edge splitting. The green line represents results from
using node movement after mesh generation (Node Movement Effect). The
yellow line represents the results from including local reconnection in
addition to node movement. The addition of node movement made a
significant difference in improving mesh quality. This is quantified
by noticing the improvement in interior angle distribution away from the
extremes of $0$ and $180$ degrees. A further improvement is seen from
using local reconnection. The yellow line shows further improvement in
the interior angle distribution with the vast majority of interior
angles being between $35$ and $85$ degrees.

\begin{figure}[h!]
  \begin{center}
  \includegraphics[width=90mm]{Figures/RDvsMeshSize.png}
  \caption{Output Surface $RD$ and Mesh Size vs. $tol_{RD}$}
  \label{fig_RDvsMeshSize}
  \end{center}
\end{figure}

A clear trend {\bf{Dave: A trend in what?}} can be seen in 
Fig. \ref{fig_RDvsMeshSize}: as the value of
$tol_{RD}$ is decreased, the representation deficit of the generated mesh
also decreases (blue line). Also, as with many optimization problems,
this particular geomtery (i.e., the ``peaks'' surface) and this particular 
initial mesh (i.e., which contains two triangles) have noticeable local 
minima in the search space.
Figure \ref{fig_RDvsMeshSize} clearly shows three plateaus of output
RD in the output mesh with decreasing $tol_{RD}$. These occur at
$\sim42\%, \sim35\%,$ and $\sim1\%$. The green bars depict the output
mesh size and also convey the diminishing returns of decreasing
$tol_{RD}$ too much. Only $18$ triangles are needed for the output RD to
reach $0.42$ ($42\%$). However, an additional $156$ triangles were
needed to reduce the output RD by ten nodes ($\sim 35\%$). {\bf{Dave - 
I'm lost. What do you mean by 10 nodes?  10 percent?  10 fewer nodes 
(meaning nodes) in the mesh?}}  To get much closer to another local 
minimum (possibly the global minimum) many
more triangles were required, i.e., $738$ in total to reach $\sim 1\%$ 
output RD. In each case, the optimal substructure of the problem is 
apparent with the value of $tol_{RD}$ being greater than the 
representation deficit of the generated mesh.


\section{Conclusions}
{\bf{Dave:  Please write some conclusions here.}}

{\bf{Dave:  The future work part of this section starts here.}} There are 
several possible directions for future work.  One avenue for future work 
is to further improve the optimization algorithm in the following two 
ways.  First, optimization could be employed in the determination of which 
mesh operation to perform at each stage of the optimization procedure.  
Second, the objective function could be modified in order to generate 
surface meshes that simultaneously minimize the representation deficit and 
optimize the mesh quality.  Another possibility for future research would 
be to extend the surface mesh generation algorithm to handle the volume 
mesh generation case so that a hierarchical approach for mesh generation 
(i.e., edge grid~\cite{mclaurin13} $\mapsto$ surface 
mesh~\cite{mclaurin14} $\mapsto$ volume mesh) could be employed. This 
would allow for optimization of each stage of the mesh generation 
procedure.  Finally, parallelization of the algorithm would allow for 
optimal generation of very large surface meshes such as those required by 
{\bf{Dave:  Cite a ``killer'' engineering application here.}} 
applications.


\iffalse
% For one-column wide figures use
\begin{figure}
% Use the relevant command to insert your figure file.
% For example, with the graphicx package use
  \includegraphics{example.eps}
% figure caption is below the figure
\caption{Please write your figure caption here}
\label{fig:1}       % Give a unique label
\end{figure}
%
% For two-column wide figures use
\begin{figure*}
% Use the relevant command to insert your figure file.
% For example, with the graphicx package use
  \includegraphics[width=0.75\textwidth]{example.eps}
% figure caption is below the figure
\caption{Please write your figure caption here}
\label{fig:2}       % Give a unique label
\end{figure*}
%
% For tables use
\begin{table}
% table caption is above the table
\caption{Please write your table caption here}
\label{tab:1}       % Give a unique label
% For LaTeX tables use
\begin{tabular}{lll}
\hline\noalign{\smallskip}
first & second & third  \\
\noalign{\smallskip}\hline\noalign{\smallskip}
number & number & number \\
number & number & number \\
\noalign{\smallskip}\hline
\end{tabular}
\end{table}
\fi

%\begin{acknowledgements}
%If you'd like to thank anyone, place your comments here
%and remove the percent signs.
%\end{acknowledgements}

% BibTeX users please use one of
%\bibliographystyle{spbasic}      % basic style, author-year citations
\bibliographystyle{spmpsci}      % mathematics and physical sciences
%\bibliographystyle{spphys}       % APS-like style for physics
\bibliography{References}   % name your BibTeX data base

\end{document}
% end of file template.tex

