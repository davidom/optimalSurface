% Discrete Curvature Approximation
\section{Discrete Curvature Approximation [UNDER HEAVY CONSTRUCTION]}
The concept of ``deviation'' as defined above is relatively
straightforward and intuitive. However, another related way of
describing ``how well'' a \textit{discretization} represents a
\textit{surface} is the degree to which the discrete representation
approximates curvature -- where curvature is defined as the amount of
``bend'' in a \textit{curve} or surface, or ``how much'' a
\textit{curve} or surface ``differs'' from a straight line or plane
(words in quotes are subject to gradation). First, however, curvature
must be defined in such a way that a discrete approximation is
meaningful and appropriate. In relevant literature, there are many ways
to estimate curvature \cite{hermann07}. Some of it bears repeating,
because it is germane to what is being discussed here: Consider the
following surface, $S$, embedded in ${\mathbb R}^3$, at point P. The
concept of an osculating circle does not generalize to higher dimensions
and therefore cannot be used here as an approximation of curvature.
Instead of determining a three-dimensional analog to an osculating
circle for $S$, an osculating sphere can be defined for a triangle, $T$,
contained in a discretization, $D$, which is approximating $S$.

A value of curvature can be calculated for each triangle in the
\textit{discretization} by considering the corresponding osculating
sphere for a given triangle. The osculating sphere here (sphere,
\Cref{fig:OsculatingCircle}) can be approximated by considering the
circumscribed sphere (circumsphere) \cite{casey1888} defined by the
three points of the triangle, $P_0$, $P_1$, and $P_2$, and a point, P on
the surface located within the triangle in $(u,v)$ space.
(\Cref{fig:CircumscribedCircle}) --- the radius of the circumsphere will
be referred to as the discrete radius of curvature.  

[REWORD BELOW HERE FOR A SPHERE]
Consider a circular segment, which represents the \textit{curve} $s$.
The corresponding chord, which represents a segment in the
\textit{discretization}--$a$, and saggitta--which represents the
``deviation'' of the segment away from the \textit{curve} $h$ is shown
in \Cref{fig:CircleGeometry}.  

[THEOREM AND PROOF ON OSCULATING SPHERE]

Given the aforementioned proof, if the discrete curvature radius is
divided by the length of the corresponding segment on the
\textit{curve}, then it’s value approaches zero. This is a
scale-independent measure that converges to a computer representable
number as the \textit{discretization} approaches the length of the
\textit{curve}. The parameter is known as the curvature ratio. It is an
intuitive measure that relates ``how far'' the \textit{curve} deviates
from the segment that is representing it as the ratio of those lengths.
Consider a point on a \textit{curve} between two endpoints of a segment
of a \textit{discretization}, as seen in \Cref{ref:CurvatureRatio}. The
length of the segment, $L_i$, is the distance from $P_0$ to $P_1$. The
perpendicular distance between the point on the \textit{curve} between
the end points and the segment is $L_c$. The three points define a
curvature ratio through the ratio of $L_c$ to $L_i$.


 As shown in \cite{mclaurin12}, deviation-based methods are
intuitive and straightforward to implement. However, drawbacks include
the fundamental lack of consistently being able to indicate
discretization accuracy for curves that are not ``well-behaved'' between
discrete segments.
