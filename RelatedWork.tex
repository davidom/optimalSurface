In general, grid generation is a name for any process that creates a
grid. For example, the advancing-front algorithm advances boundaries
into space to generate a grid \cite{tristrano98, diaz-morcillo98}[CITE].
Other methods generate grids from iterative refinement or enrichment
from initial, coarse configurations \cite{marcum98, marcum00,
shewchuk02}[CITE]. Usually the benchmark for separating the two
methods, generation and refinement, are the prioritization of grid
quality and grid accuracy (both of these issues will be addressed
later).

Traditionally, surface grid generation processes produce good quality
grids from the combination of geometric growth rates and smoothing.
However, the process requires input and if the input, or starting place,
is not appropriate, then the geometry can be under and/or over sampled
for the intended use. That fact is not an indictment of the grid
generation process, but instead implies that the final grid is heavily
dependent on the inputs. In addition, if some way of controlling the
point spacing in the middle of the surface (away from the boundaries) is
not present, then more points could be wasted/omitted in an attempt to
accurately represent the geometry.

Other efforts have gone into creating a locally or globally optimal
surface grid. Many names have been assigned to the particular task, but
the underlying goal is very similar --- represent a surface as
accurately as possible---whatever that means for each application. For
example, [CITE, EXPLAIN, CITE, EXPLAIN, CITE, EXPLAIN]
