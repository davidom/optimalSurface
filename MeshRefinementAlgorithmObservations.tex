\subsection{Observations}
\subsubsection{Nodal Smoothing}
In practice it was observed that nodes will often move toward each other
when there is a ``local minimum'' nearby ({\it race-condition}).
However, once the two nodes get sufficiently near to each other one will
inevitably get nearer to the local minimum first and the other node will
be ``pushed'' away.

\subsubsection{Edge Flipping}
Another phenomenon that was noticed was that during triangle splitting
and smoothing the nodes would often be repositioned near edges. This
usually meant that the nearby minimum was either on the edge or on the
other topological side of the edge. Without some sort of intervention,
this would lead to nearly degenerate geometry being formed. In order to
prevent geometry which could cause numerical robustness issues, when a
node neared an edge during node movement the edge was flipped. This
allowed the node to move unobstructed by the newly-reconnected edge.

The criteria for determining if a node is ``too close'' to an edge is
based on the size of the pattern[CLARIFY]. If the minimum value for the
objective function is on or across an edge, then the edge is flipped and
the process is repeated with the new topology.
