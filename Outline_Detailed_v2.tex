\documentclass[11pt]{article}

\usepackage[top=1in, bottom=1.5in, left=1in, right=1in]{geometry}
\usepackage{amssymb}

\begin{document}
\section{Introduction}

\section{Related Work}

\section{Nomenclature and Definitions}
\subsection{Parametric Surface}
I need a parametric surface so that I can maintain a valid topology.
Projection is not required since the topology gives me uv bounds. The
mesh generation is going on in parametric space. There is an
optimization function for creating the mesh parametric space. Parametric
space is also great for optimization since you can write the topology
and optimization function constraints in uv space. Mapping a
discretization to a surface in general involved the determination of
"what" an edge means on the surface---whether it's a geodesic line, or
what...

Consider an orientable, parameterized surface, $\vec{S}\left(u,v\right)
  : {\mathbb R}^2 \rightarrow {\mathbb R}^3$.
\subsection{Discretization}
Consider a discretization, $D$, defined on $S$, comprised of $n_t$
points, $P_i \in \left\{p_1,...,p_{n_t} \right\}$, and a
non-overlapping, non-degenerate, consistently oriented, triangular
connectivity $T$. Each triangle in $T$, $t_i$, is defined by points
$\left(p_j, p_k, p_m\right)$, and edges $\left(e_n, e_o,
e_p\right)$---where the edges are defined (for $t_i$) as $e_n
\left(p_j, p_k\right)$, $e_o \left(p_k, p_m\right)$, $e_p \left(p_m,
p_j\right)$. For the purposes of defining an edge in this work, the
ordering of the nodes does not matter. However, the ordering is used to
maintain consistent orientation during topological operations and to
define constraints in the developed optimization strategy.

Each point , $p_i$, in $D$ is defined at a $\left(u,v\right)$
coordinate, and edge edge is a straight line in both parametric space
and in ${\mathbb R}^3$. In the absence of a parameterization the mapping
of an edge in ${\mathbb R}^3$ to a curve (possibly a straight line, but
not guaranteed) on a surface is an ambiguous task which is outside
the scope of this work. Additionally, topological constraints for
optimization, which will be detailed later, would be impossible in
general to define outside of a parameterized, planar space.[MAYBE MORE
ON THIS AND MAYBE MOVE THIS TO ANOTHER SECTION OR REORDER TOPICS]

\subsubsection{Trias, Edges}
\subsubsection{Toplogy}
By ordering each the connectivity for each triangle in $T$ such that it
has a positive normal, topological constraints can be defined on the
optimization problem.
\subsection{Discretization Error}
\subsection{Representation Deficit}
Representation deficit is defined for an operation:
\subsubsection{Triangle Split}
The representation deficit for a triangle split is defined here as the
difference between the area original triangle and the combined area of
the three new triangles---where the area is calculated in ${\mathbb
R}^3$.
\subsubsection{Edge Split}
The representation for an edge split must consider both the change in
edge length as well as the change in area that can occur when an edge is
split. But is this necessary to consider the edge-length since the
purpose is refine such that the surface area of $D$ approaches the
surface area of $S$? The answer here is no, only the area change is
considered between the previous configuration and the ``new''
configuration.
\subsubsection{Edge Flipping}
Is it possible to have an optimal topology? If so, then it should be
reachable through repeated edge flipping. Therefore, we should be able
to write some sort of formulation for a pair of triangles and go from
there? [SUZANNE] Comment: for the sake of not doing any more coding, if
your ethics can let you come to the conclusion that this is either not
possible or not practical then this would fit into the
logic/justification in the next section.
\subsubsection{Nodal Smoothing}
For a node, $n_i$, the representation deficit can be defined only by
comparing it to the node when located at another point in space. Here
the comparison is bound by limiting the range of comparison within the
edge-hull topologically adjacent to $n_i$. This is functionally
equivalent to smoothing to maximize surface area.

\section{Optimal Substucture vs Global Optimization}
[NEED TO COME TO THE CONCLUSION HERE THAT GLOBAL OPTIMIZATION IS NOT
PRACTICAL] [COMMENT ON HOW ITERATIVE REFINEMENT IS USED HERE, NOT
SETTING UP SOME GLOBAL EQUATIONS]
\subsection{Local Optimization via Pattern Matching}
[Suzanne] I used a simple "cross" pattern (up down left right center)
and sized it based on the local geometry so that it could be moved
inside of the triangles/hulls many times before encountering the
boundary.

\section{Representation Deficit as Local/Iterative Optimization}
Let us develop the definition of a local optimization problem using the
concept of representation deficit as an objective function. First
consider all of the variables: the number of points ($n_p$), the
optimized topology/connectivity ($T_o$), and the position/location of
the points ($\vec{N_i}$).
\subsection{Fixed $n_p$, Fixed $T_o$, Dyanamic $\vec{N_i}$}
To start, let us define the problem as simply as possible: a fixed
number of points, a fixed topology, and the points have a variable
position.  The local optimization problem is defined as follows: [For
each node, $n_i$, determine the surface area of the topologically
adjacent triangles, $A_0$. Calculate the objective function for each of
the four candidate new positions in the pattern, $\left(O_{up},
O_{down}, O_{left}, O_{right}\right)$. Determine the minimum value from
the objective functions calculations, $O_{min}$. Compare $O_{min}$ to
the representation deficit change limit. Move if appropriate]
[DEFINITION OF OPTIMIZATION PROBLEMS]
[HOW TO FIT IN THE EDGE FLIPPING AS AN OPTIMIZATION PROBLEM? I CAN
JUSTIFY IT VIA HAND-WAVING, BUT NEED SOME WAY TO EXPLAIN IT BETTER]
\subsubsection{Race Condition}
In practice it was observed that nodes will often move toward each other
when there is a ``local minimum'' nearby. Here this phenomenon is called
a {\it race-condition}. However, once the two nodes get sufficiently
near to each other one will inevitably get to the local minimum first
and the other node will be ``pushed'' the away.
\subsection{Fixed $n_p$, Dynamic $T_o$, Dyanamic $\vec{N_i}$}
\subsubsection{Edge Flipping}
Another phenomenon that was noticed was that during the smoothing the
nodes would often be repositioned near edges. This usually meant that
the nearby minimum was on the edge or on the other topological side of
the edge. Without some sort of intervention, this would lead to nearly
degenerate geometry being formed. In order to prevent geometry which
could cause numerical robustness issues, when a node neared an edge the
edge was flipped. This allowed the node to move unobstructed by the
newly-reconnected edge.

The criteria for determining if a node is ``too close'' to an edge is
based on the pattern. If the minimum value for the objective function is
on or across an edge, then the edge is flipped and the process is
repeated with the new topology.

\subsection{Dynamic $n_p$, Dynamic $T_o$, Fixed $\vec{N_i}$}
This formulation is not well-posed in that without some sort of limit on
the number of nodes to add to the discretization {\it the} optimal
triangulation has infinite points. An infinitely refined surface would
have no discretization error. However, there is a limiting term that
ensures that infinite refinement will not occur: representation-deficit
limit [RENAME?]. This method of optimization takes advantage of the
optimal substructure associated with this optimization problem. Each
triangle is considered and the optimal nodal position is calculated
using the aforementioned pattern matching. However, there are some
limits on where these points can reside. If the pattern-matching leads
to a nodal position is too close to an edge then it is rejected.

\subsection{Dynamic $n_p$, Dynamic $T_o$, Dynamic $\vec{N_i}$}
This formulation combines all of the above modes by iteratively
combining representation-deficit driven refinement with nodal smoothing
and local reconnection.

\section{Error Bounds}
\section{Experimental Results}
\section{Conclusions}
\section{Acknowledgements}
\end{document}
