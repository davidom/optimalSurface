\documentclass[11pt]{article}

\usepackage[top=2in, bottom=1.5in, left=1in, right=1in]{geometry}
\usepackage{amssymb}

\begin{document}
\section{Introduction}
\section{Related Work}
\section{Nomenclature and Definitions}
\subsection{Parametric Surface}
I need a parametric surface so that I can maintain a valid topology.
Projection is not required since the topology gives me uv bounds. The
mesh generation is going on in parametric space. There is an
optimization function for creating the mesh parametric space. Parametric
space is also great for optimization since you can write the topology
and optimization function constraints in uv space. Mapping a
discretization to a surface in general involved the determination of
"what" an edge means on the surface---whether it's a geodesic line, or
what...

Consider an orientable, parameterized surface, $\vec{S}\left(u,v\right)
  : {\mathbb R}^2 \rightarrow {\mathbb R}^3$.
\subsection{Discretization}
Consider a discretization, $D$, defined on $S$, comprised of $n_t$
points, $P_i \in \left\{p_1,...,p_{n_t} \right\}$, and a
non-overlapping, non-degenerate, consistently oriented, triangular
connectivity $T$. Each triangle in $T$, $t_i$, is defined by points
$\left(p_j, p_k, p_m\right)$, and edges $\left(e_n, e_o,
e_p\right)$---where the edges are defined (for $t_i$) as $e_n
\left(p_j, p_k\right)$, $e_o \left(p_k, p_m\right)$, $e_p \left(p_m,
p_j\right)$. For the purposes of defining an edge in this work, the
ordering of the nodes does not matter. However, the ordering is used to
maintain consistent orientation during topological operations and to
define constraints in the developed optimization strategy.

Each point , $p_i$, in $D$ is defined at a $\left(u,v\right)$
coordinate, and edge edge is a straight line in both parametric space
and in ${\mathbb R}^3$. In the absence of a parameterization the mapping
of an edge in ${\mathbb R}^3$ to a curve (possibly a straight line, but
not guaranteed) on a surface is an ambiguous task which is outside
the scope of this work. Additionally, topological constraints for
optimization, which will be detailed later, would be impossible in
general to define outside of a parameterized, planar space.[MAYBE MORE
ON THIS AND MAYBE MOVE THIS TO ANOTHER SECTION OR REORDER TOPICS]

\subsubsection{Trias, Edges}
\subsubsection{Toplogy}
By ordering each the connectivity for each triangle in $T$ such that it
has a positive normal, topological constraints can be defined on the
optimization problem.
\subsection{Discretization Error}
\subsection{Representation Deficit}
Representation deficit is defined for an operation:
\subsubsection{Triangle Split}
The representation deficit for a triangle split is defined here as the
difference between the area original triangle and the combined area of
the three new triangles---where the area is calculated in ${\mathbb
R}^3$.
\subsubsection{Edge Split}
The representation for an edge split must consider both the change in
edge length as well as the change in area that can occur when an edge is
split. But is this necessary to consider the edge-length since the
purpose is refine such that the surface area of $D$ approaches the
surface area of $S$? The answer here is no, only the area change is
considered between the previous configuration and the ``new''
configuration.
\subsubsection{Nodal Smoothing}
For a node, $n_i$, the representation deficit can be defined only by comparing
it to the node when located at another point in space. Here the
comparison is bound by limiting the range of comparison within the edge-hull
topologically adjacent to $n_i$. This is functionally equivalent to
smoothing to maximize surface area.
\section{Optimal Substucture vs Global Optimization}
\subsection{Representation Deficit as a Global Optimization Problem}
Let us develop the definition of a global optimization problem using the
concept of representation deficit as an objective function. First
consider all of the variables: the number of points ($n_p$), the
optimized topology/connectivity ($T_o$), and the position/location of
the points ($\vec{N_i}$).
\subsubsection{Fixed $n_p$, Fixed $T_o$}
To start, let us define the problem as simply
as possible: a fixed number of points, a fixed topology, and the points
have a variable position.  The optimization problem is defined as
follows:
[DEFINITION OF OPTIMIZATION PROBLEMS]
[HOW TO FIT IN THE EDGE FLIPPING AS AN OPTIMIZATION PROBLEM? I CAN
JUSTIFY IT VIA HAND-WAVING, BUT NEED SOME WAY TO EXPLAIN IT BETTER]
\section{Clarification on Mesh Refinement}
\section{Refinement via Representation Deficit}
\subsection{Triangle Split}
\subsection{Edge Split}
\subsection{Node Smoothing}
\subsection{Notes on Edge-Swapping}
\section{Error Bounds}
\section{Experimental Results}
\section{Conclusions}
\section{Acknowledgements}
\end{document}
