The following algorithm is used to generate an optimal triangulation
based on representation deficit refinement.  The heuristic schedule of 
mesh operations to be performed was developed based upon our observations 
on the performance of various combinations of the mesh operations.
Note that heuristic schedules have been used for scheduling geometric and 
topological mesh operations in order to improve mesh element quality; one 
of the most advanced heuristic schedules for used for this purpose can be 
found in~\cite{tet_aggressive}. Alg.(\ref{alg_IterativeRefinement}) is
performed until there are no longer any triangles in the mesh that can
be refined.

\subsection{Algorithm Specification}
\begin{algorithm}[H]
\begin{algorithmic}
\caption{Find optimal location for triangle splitting}
\Procedure{findOptimalNode}{$t_i$} \Comment{for Triangle}
  \State $pattern \gets sizePattern\left( t_i \right)$
  \State $pattern \gets calculateTrianglePatternValues\left( t_i \right)$
  \State $minValue \gets determineMinPatternValue\left( pattern \right)$
  \While{$newTriangulationIsValid\left( minValue \right)$}
    \State $pattern \gets shiftPattern\left( minValue \right)$
    \State $pattern \gets calculateTrianglePatternValues\left( t_i \right)$
    \State $minValue \gets determineMinPatternValue\left( pattern \right)$
  \EndWhile
  \State $p_i \gets currentPatternValue\left( pattern \right)$
  \State $return \gets \left( p_i, minValue \right)$
\EndProcedure
\end{algorithmic}
\end{algorithm}

\begin{algorithm}[H]
\begin{algorithmic}
\caption{Find optimal location for edge splitting}
\Procedure{findOptimalNode}{$e_i$} \Comment{for Edge}
  \State $pattern \gets sizePattern\left( e_i \right)$
  \State $pattern \gets calculateEdgePatternValues\left( e_i \right)$
  \State $minValue \gets determineMinPatternValue\left( pattern \right)$
  \While{$newTriangulationIsValid\left( minValue \right)$}
    \State $pattern \gets shiftPattern\left( minValue \right)$
    \State $pattern \gets calculateEdgePatternValues\left( e_i \right)$
    \State $minValue \gets determineMinPatternValue\left( pattern \right)$
  \EndWhile
  \State $p_i \gets currentPatternValue\left( pattern \right)$
  \State $return \gets \left( p_i, minValue \right)$
\EndProcedure
\end{algorithmic}
\end{algorithm}

\begin{algorithm}[H]
\caption{Iterative Refinement}
\label{alg_IterativeRefinement}
\begin{algorithmic}
\Procedure{TriangleRefine}{$T$, $tol_{RD}$} \Comment{Refine the triangulation, $T$}
  \For{each triangle, $t_i$}
    \State $\left( p_i, minValue \right) \gets findOptimalNode\left( t_i \right)$
    \If{$tooCloseToEdge\left( p_i, t_i \right)$}
      \State $e_i \gets findNearEdge\left( p_i, t_i \right)$
      \State $\left( p_i, minValue \right) \gets findOptimalNode\left( e_i \right)$
      \State $operation \gets EdgeSplit\left( e_i, p_i \right)$
    \Else
      \State $operation \gets triangleSplit\left( t_i, p_i \right)$
    \EndIf
    \If{$minValue < tol_{RD}$}
      \State perform $operation$
    \EndIf
  \EndFor
\EndProcedure
\end{algorithmic}
\end{algorithm}

\begin{algorithm}
\begin{algorithmic}
\caption{Find the optimal location for node relocation}
\label{alg_NodeSmoothing}
\Procedure{findOptimalNode}{$n_i$} \Comment{for Node}
  \State $pattern \gets sizePattern\left( n_i \right)$
  \State $pattern \gets calculateNodePatternValues\left( n_i \right)$
  \State $minValue \gets determineMinPatternValue\left( pattern \right)$
  \While{$newTriangulationIsValid\left( minValue \right)$}
    \State $pattern \gets shiftPattern\left( minValue \right)$
    \State $pattern \gets calculateNodePatternValues\left( n_i \right)$
    \State $minValue \gets determineMinPatternValue\left( pattern \right)$
  \EndWhile
  \State $p_i \gets currentPatternValue\left( pattern \right)$
  \State $return \gets \left( p_i, minValue \right)$
\EndProcedure
\end{algorithmic}
\end{algorithm}

\begin{algorithm}[H]
\caption{Nodal Movement}
\begin{algorithmic}
\Procedure{NodalSmooth}{$T$, $tol_{RD}$} \Comment{Smooth Node Positions}
  \For{each node, $n_i$}
    \State $pattern \gets sizePattern\left( e_i \right)$
    \State $pattern \gets calculateNodePatternValues\left( n_i \right)$
    \State $\left( p_i, minValue \right) \gets findOptimalNode\left( e_i \right)$
  \EndFor
  \If{$minValue < tol_{RD}$}
    \State relocate $n_i$ to $p_i$
  \EndIf
\EndProcedure
\end{algorithmic}
\end{algorithm}

Another phenomenon that was noticed was that during
triangle splitting and node movement, a node would often be repositioned
near edges. This usually meant that the nearby local minimum was either
on the edge or on the other topological side of the edge. Without some
sort of intervention, this would lead to nearly degenerate geometry
being formed. In order to prevent geometry which could cause numerical
robustness issues, when a node neared an edge during node movement the
edge was flipped. This allowed the node to move unobstructed by the
newly-reconnected edge.

The criteria for determining if a node is ``too close'' to an edge is
based on the size of the pattern. Once the node gets within $1\%$ of the
average edge length, which equates to $10$ times the pattern size, it is
considered ``too close''. There is a balance here between preventing the
creation of degenerate geometry and allowing the pattern search to find
{\it and then use} an optimum value. The above heuristics were found to
be a good compromise and will be shown to be effective. If the minimum
value for the objective function is on or across an edge, then the edge
is flipped and the process is repeated with the new topology.

It was also observed that nodes will often move toward
each other when there is a ``local minimum'' nearby ({\it
race-condition}).  However, once the two nodes get sufficiently near to
each other, one will inevitably get nearer to the local minimum first
and the other node will be ``pushed'' away.
