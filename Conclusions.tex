In this work, the concept of optimal mesh refinement was explored. This
was done through the formulation of optimization problems which, when
solved, yield optimal triangular surface meshes based on minimizing the
{\it representation deficit}.  Optimization problems were developed for
three fundamental mesh operations, i.e., triangle split, edge split, and
node movement.  The optimization problems were solved using a local,
constrained numerical optimization approach similar to a pattern search.
A heuristic schedule of mesh operations to be performed was developed
based upon the authors' observations on the performance of various
combinations of the operations. The chosen combination addresses the
main drawback of the developed algorithm: element quality is not
considered during mesh generation. Results were given showing this
methodology to be an effective method for mesh refinement. The most
effective method was found to be a combination of the above three
fundamental mesh operations and Delaunay local reconnection. The
representation is also efficient in that the interior of the domain was
not overly refined in order to achieve the desired surface
representation.

%\subsection{Future Work}
There are several possible directions for future work.  One avenue for
future work is to further improve the optimization algorithm in the
following two ways.  First, optimization could be employed in the
determination of which mesh operation to perform at each stage of the
optimization procedure.  Second, the objective function could be
modified in order to generate surface meshes that simultaneously
minimize the representation deficit and optimize the mesh quality.
Another possibility for future research would be to extend the surface
mesh generation algorithm to handle the volume mesh generation case so
that a hierarchical approach for mesh generation (i.e., edge
grid~\cite{mclaurin13} $\mapsto$ surface mesh~\cite{mclaurin14}
$\mapsto$ volume mesh) could be employed. This would allow for
optimization of each stage of the mesh generation procedure.  Finally,
parallelization of the algorithm would allow for optimal generation of
very large surface meshes such as those required for computational
simulations of turbulent fluid flow~\cite{turbulence} and coastal
ocean modeling~\cite{coastal_ocean_modeling}.
