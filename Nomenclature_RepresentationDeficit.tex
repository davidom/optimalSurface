The concept of representation deficit was discussed in \cite{mclaurin13} in the
context of curve discretizations. Here, the concept will be extended to
two dimensions. Representation deficit is defined for an operation:

[SCALE INDEPENDENCE]

\subsubsection{Triangle}
A triangle is a planar object, and is representing a possibly non-planar
portion of a surface. Any triangle, with area $A_T$, in the
discretization can have at most the same area as the portion of the
surface which it represents, $A_S$. Therefore, the representation
deficit for a triangle, $RD_T$, can be defined as $RD_T = \frac{A_S -
A_T}{A_T}$. Note that the areas are calculated in ${\mathbb R}^3$. The
difference in surface area, $\left(A_S - A_T\right)$, is normalized by
$A_T$ so that the result is scale independent. Also, this is a
representation {\it deficit} since $A_S \ge A_T$ is always true.

In order to apply the above definition of representation deficit to a
mesh generation, a replacement for $A_S$ must be determined since the
area of the surface represented by $A_S$ is not always able to be
determined --- or, most often, the area calculation is impractical.
Generally, in order to reduce the representation deficit for a triangle
the triangle is split by inserting an interior point. Any point that is
inserted into the interior of the triangle would decrease the
representation deficit --- or at worst it will remain the same. However,
the determination of where to split the triangle should be done in such
a way that the representation deficit is minimized. This strategy of
refinement, refining each triangle in such a way that the representation
deficit is locally minimized, would take advantage of the optimal
substructure the discrete topology.

The process of determining where to split a triangle is defined here by
a locally optimizing an objective function defined for a triangle: Let
$S(u,v)$ be a parameterized surface, $D$ be be a discretization of the
surface, and $T$ be a triangle (included in $D$) defined by an ordered
tuple of nodes, $\left(n_i, n_j, n_k\right)$. These nodes are ordered
such that the triangle normal is positive. Specifically, if $\vec{V_0} =
\left\{n_j - n_i \right\}$ and $\vec{V_1} = \left\{n_k - n_i\right\}$
then the triangle normal, $N_T = \vec{V_0} \times \vec{V_1}$, is
positive. Note that a two-dimensional cross-product (or two-dimensional
curl) is a scalar quantity. Additionally, let a node on the interior of
$T$ be defined as $n_O$. The four nodes, $n_i, n_j, n_k, and n_O$ define
three triangles, $T_i\left(n_i,n_j,n_0\right), T_j\left(n_j, n_k,
n_0\right), T_k\left(n_k, n_i, n_0\right)$. If $A(T)$ is a function that
calculates the area of a triangle in $\left(x,y,z\right)$ space, then
the optimization problem for finding the optimal position for $n_0$ in
$T$ can be stated as:

\begin{eqnarray*}
\begin{array}{rcl}
\underset{n_O}{\text{minimize}} \ O(T) & = & -\left(A\left(T_i\right) + A\left(T_j\right) + A\left(T_k\right) \right) \\
\text{subject to} \ N_{T_i} & > & 0 \\
N_{T_j} & > & 0 \\ 
N_{T_k} & > & 0 \\
\end{array}
\end{eqnarray*}

It should be noted that this definition of representation deficit would be difficult to derive or express for a topological entity other than a triangle. This is due to the inherent ambiguity in the definition of not only the surface area, but also the surface representation of (possibly) non-planar elements, e.g. non-planar, or skew, quadrilateral. [MORE POSSIBLY]

\subsubsection{Edge}
An edge in parametric space represents a (possible) curve on the
surface. Any edge, with length $L_E$, can have at most the same length
as the portion of the surface which it represents, $L_S$. Therefore, the
representation deficit for an edge, $RD_E$, can be defined as $RD_E =
\frac{L_S - L_E}{L_E}$. Note that the lengths are calculated in
${\mathbb R}^3$. The difference in length, $\left(L_S - L_E\right)$, is
normalized by $L_E$ so that the result is scale independent. Also, this
is a representation {\it deficit} since $L_S \ge L_E$ is always true;

In order to apply the above definition of representation deficit to mesh
generation, a replacement for $L_S$ must be determined since the area of
the surface represented by $L_S$ is not always able to be determined --
or, most often, the arc-length calculation is impractical.
Generally, in order to reduce the representation deficit for an edge 
the edge is split by inserting an interior point. Any point that is
inserted into the interior of the edge would decrease the
representation deficit --- or at worst it will remain the same. However,
the determination of where to split the edge should be done in such
a way that the representation deficit is minimized. This strategy of
refinement, refining each edge in such a way that the representation
deficit is locally minimized, would take advantage of the optimal
substructure the discrete topology. A method for generating a locally
optimal edge split is detailed in \cite{mclaurin12,mclaurin13}[OTHERS].

In addition, the fact that and edge split will change the surface area
of the discretization should be considered. Since the overall
justification of this work is reduce the {\it area} representation
deficit of a discretization, the representation deficit will not be
defined for an edge but for an edge-split. Therefore, for a given edge,
only the edge-split that minimizes representation deficit is considered.
. The process of determining where to split a triangle is defined here
by locally optimizing an objective function defined for an edge-split.
Let $E\left(n_i,n_j\right)$ be an edge in $D$ that is shared
topologically by two triangles, $T_i\left(n_i,n_j,n_k\right)$, and
$T_j\left(n_i,n_l,n_j\right)$. Additionally, let a node on the interior
of the edge be defined as $n_O$. The five nodes, $n_i,n_j,n_k,n_l,n_O$
define four triangles, $T_a\left(n_i,n_O,n_k\right), T_b\left(n_O,n_j,
n_k\right), T_c\left(n_i,n_l,n_O\right), T_d\left(n_O,n_l,n_j\right)$.
The optimization problem for finding the optimal position for $n_O$ on
$E$ can be stated as:

\begin{eqnarray*}
\begin{array}{rcl}
\underset{n_O}{\text{minimize}} \ O(E) & = & -\left(A\left(T_a\right) + A\left(T_b\right) + A\left(T_c\right) A\left(T_d\right) \right) \\
\text{subject to} \ N_{T_a} & > & 0 \\
N_{T_b} & > & 0 \\ 
N_{T_c} & > & 0 \\
N_{T_d} & > & 0 \\
\end{array}
\end{eqnarray*}

\subsubsection{Edge Flipping}
Is it possible to have an optimal topology? If so, then it should be
reachable through repeated edge flipping. Therefore, we should be able
to write some sort of formulation for a pair of triangles and go from
there? [SUZANNE] Comment: for the sake of not doing any more coding, if
your ethics can let you come to the conclusion that this is either not
possible or not practical then this would fit into the
logic/justification in the next section.

\subsubsection{Nodal Smoothing}
For a node, $N_i$, the representation deficit can be defined only by
comparing it to the node when located at another point in space. Here
the comparison is bound by limiting the range of comparison within the
edge-hull topologically adjacent to $N_i$. Formally, let $N_i$ be a node
in $D$ that is shared topologically by $n$ triangles, where $n$ is the
face-valence of $N_i$. The optimization for finding the optimal position
for $N_i$ can be stated as:

\begin{eqnarray*}
\begin{array}{rcl}
\underset{N_O}{\text{minimize}} \ O(N) & = &
-\sum{_{j=1}^{n_t-1}A\left(T_j\right)} \\
\text{subject to} \ N_{T_1} & > & 0 \\
N_{T_2} & > & 0 \\ 
N_{T_3} & > & 0 \\
& \vdots & \\
N_{T_{n-1}} & > & 0 \\ 
N_{T_n} & > & 0
\end{array}
\end{eqnarray*}

