The accuracy, or discretization error, of a piecewise, linear
representation of an analytical surface in ${\mathbb R}^3$ can be
defined in many ways depending on the intended application. The error
associated with the discretization is discussed in terms of the
deviation from the surface---most often quantified by calculating or
approximating the distance from the surface for each linear entity
(triangles and edges) in the discretization.  Another way of quantifying
the error associated with a discretization is to consider how well
it approximates the surface area of the surface it represents.  In
general, the surface area is not known {\it a priori}; however, for some
surfaces, depending on the underlying representation, it can be 
calculated exactly via computation of a double integral or estimated using 
numerical cubature.  In other cases, such as when the surface is very 
irregular or rough, it is not always possible to assign a surface area
without a generalization of the concept from geometric measure 
theory~\cite{gmt}.  Two goals of the developed 
method is that it be general, i.e., it 
should be independent of the underlying geometric representation, and 
practical.  Therefore, a method requiring the surface area of the 
underlying geometry violates the aforementioned concept of 
generality and restricts the applications for which the proposed method 
can be applied and hence is impractical.  Another way of 
determining/generating a surface grid based on surface area is needed. 
This process is detailed later.

The concept of deviation defined above is relatively straightforward and
intuitive. However, another related way of describing how well a
\textit{discretization} represents a \textit{surface} is the degree to
which the discrete representation approximates curvature. First,
however, curvature must be defined in such a way that a discrete
approximation is meaningful and appropriate. In relevant literature,
there are many ways to estimate surface curvature \cite{hermann07}. Some
of it bears repeating, because it is germane to what is being discussed
here: Consider a surface, $S$, embedded in ${\mathbb R}^3$, at point P.
The concept of an osculating circle \cite{weissteineOsculatingCircle}
does not generalize well to higher dimensions and therefore cannot be
used here as an approximation of curvature. Generally, a (hyper)surface
does not have a (hyper)sphere that approximates curvature, but a
(hyper)ellipsoid. This is because a surface can have multiple values and
directions of curvature at a point. Here, an osculating ellipsoid could
be constructed by considering the principal curvature directions and
surface normal at a given point on a surface. These three vectors are
orthogonal and could be used to define the semi-axes of an ellipse. If
some way could be found to determine a scale of the semi-axis associated
with the surface normal (which is a unit-vector), this ellipse would be
a good approximation of the surface in the direction of the principal
curvatures. However, this requires the computation or approximation of
derivatives---which we have ruled out here.

Instead of determining a three-dimensional analog to an osculating
circle for $S$, an osculating sphere could be defined for a triangle,
$T$. A value of curvature can be calculated for each triangle in the $D$
by considering the corresponding osculating sphere for $T$. The
osculating sphere here (sphere,[FIGURE]) can be approximated by
considering the circumscribed sphere (circumsphere) \cite{casey1888}
defined by the three points of the triangle, $P_0$, $P_1$, and $P_2$,
and a point, $P$ on the surface located within the triangle in $(u,v)$
space. The radius of the circumsphere will be referred to as the
discrete radius of curvature, $R_D$. This argument for curvature
approximation is a two-dimensional analog to the one in
\cite{mclaurin12} for approximating curvature along a curve. In that
work, the authors showed that for an edge on a curve, as the radius of
curvature of the osculating circle approaches infinity the arc length of
the circular segment approaches the arc length of the edge. However,
here as $R_D$ approaches infinity, the surface area of the spherical cap
does not approach the area of $T$---it approaches that of the
circumscribed circle defined by $T$.

Finally, what is most appropriate is to consider the surface area of the
tetrahedron, ${\mathbb T}$, formed by $P_0$, $P_1$,$P_2$, and $P$. This
is a meaningful comparison: as $R_D$ approaches infinity (as the
distance between $P$ and the plane defined by $T$ approaches $P_0$) the
surface area of ${\mathbb T}$ approaches two times the area of
$T$[ELABORATE?].  Stated another way, as $D$ is refined the surface area
of $D$ approaches that of $S$.

Surface area convergence of a discretization is a sufficient condition
for other schemes of surface grid generation/refinement. That is: if the
difference between the surface area of the surface and the sum of the
surface area of the discrete entities in the discretization approaches
zero then that is sufficient to conclude that the distance between the
discretization and surface is also approaching zero, also the dihedral
angles between segments approaches $0$ degrees [see PROOF once
definition of REPRESENTATION DEFICIT is detailed in APPENDIX]. However,
the converse of that statement is not true. The pathological case of a
highly oscillatory, low amplitude surface approximated by two triangles
[MORE DETAIL] can have a small deviation or angles between elements
but be a poor estimate for surface area.
