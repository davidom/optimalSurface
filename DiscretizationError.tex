\subsection{Discretization Error}
The accuracy, or discretization error, of a piecewise, linear representation of an analytical surface in ${\mathbb R}^3$ can be defined in many ways depending on the intended application. The error associated with with the discretization is discussed in terms of the ``deviation'' from the surface---most often quantified by calculating or approximating the distance from the surface for each linear entity (triangles and edges) in the discretization, or some [AREA RELATED ERROR]. Another way of quantifying the error associated with a discretization would be to consider how well it approximates the surface area of the surface it represents. In general the surface area is not known {\it a priori}, but depending on the underlying representation it can be calculated exactly ([EXAMPLE]) or can be estimated ([EXAMPLE]). One of the goals of the developed method was to be ``general'' in that it should be independent of the underlying geometric representation. Therefore, a method that requires the surface area of the underlying geometry violates the aforementioned concept of ``generality'' and restricts the applications for which the proposed method could be applied. Some other way of determining/generating a surface grid based on surface area is needed. This process will be detailed later.

Surface area convergence of a discretization is a sufficient condition for other schemes of surface grid generation/refinement. That is: if the difference between the surface area of the surface and the sum of the surface area of the discrete entities in the discretization approaches zero then that is sufficient to conclude that the distance between the discretization and surface is also approaching zero, also the dihedral angles between segments approaches $0$ degrees [see PROOF once definition of REPRESENTATION DEFICIT is detailed in APPENDIX]. However, the converse of that statement is not true. The pathological case of a highly oscillatory, low amplitude surface approximated by two triangles [MORE DETAIL] can have a small ``deviation'' or angles between elements but be a poor estimate for surface area.

\subsection{Discrete Curvature Approximation}
[TALK HERE ABOUT HOW AN OSCULATING ELLIPSOID COULD BE FORMED BY CONSIDERING THE PRINCIPAL CURVATURES AT A POINT BUT THIS REQUIRES DERIVATIVES. ADDITIONALLY, FOR A TRIANGLE AND A GIVEN CANDIDATE POINT, THERE IS NOT ENOUGH INFORMATION TO DEFINE AN ELLIPSOID--ONLY A SPHERE. THEREFORE SOME ANALYSIS WITH AN OSCULATING SPHERE WOULD BE INSTRUCTIVE HERE]

[EVEN IF THE DEVIATION APPROACHES ZERO THE AREA OF THE TRIANGLE DOES NOT APPROACH THAT OF THE CIRCUMCIRCLE IN THE PLANE THAT CUTS THE SPHERE. MORE APPROPRIATE WOULD BE TO CONSIDER THE TETRAHEDRON THAT IS FORMED BY THE FOUR POINTS AND HOW THE AREA OF THE OTHER THREE FACES APPROACHES THAT OF THE ORIGINAL FACE AS h->0---NO WORRIES ABOUT THE POINT BEING OUTSIDE OF THE TRIANGLE SINCE NO INVALID TOPOLOGY IS ALLOWED.]

[A DIRECT TWO DIMENSIONAL ANALOG OF A SPHERE CAP TO A CIRCULAR SEGMENT IS NOT APPROPRIATE BECAUSE OF \^||\^. TALK ABOUT HOW THE AREA OF THE CIRCUMCIRCLE IS ALWAYS AT LEAST $\frac{4}{3} * TriangleArea$, quote isoperimetric theorem for the fact that for a given family of triangles inscribed in a circle that the equilateral triangle must have the most area.]
