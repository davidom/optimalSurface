Discretization error can be defined for surface grids be comparing the
``amount'' of discrete representation to the ``amount'' of continuous
representation. Here, this means directly comparing the surface area of
the discrete surface to that of the continuous surface. For a
triangulated surface, there are two types of entities representing the
surface: line segment, and triangle. For a line segment, or edge, a
comparison should be made between the arc length of the edge and the arc
length of the portion of the surface which the edge represents. For a
triangle a comparison should be made between the area of the triangle
and the area of the portion of the surface which the triangle
represents. This difference between the surface area of the
discretization and the surface area of the underlying geometrical
representation is defined here as {\it representation-deficit}.

[PARAGRAPH ABOUT OPTIMAL SUBSTRUCTURE]

It should be noted that for an edge on a discrete surface the change in
edge-length also corresponds to a change in surface-area.Therefore, the
change in edge length and the change in surface area must be taken into
account during the refinement process. If only the edge length of the
edges is considered then it is demonstrable that the edges could have
little to no representation deficit but the triangles which are defined
by these edges are a poor representations of the underlying surface.
[FIGURE] If only the area of triangles is considered then a very
probable outcome is the formation of degenerate triangles with zero
area. [EXPLAIN FURTHER ABOUT THE REFINEMENT PROCESS NOT BEING ABLE TO
CROSS AN EDGE, OPTIMAL SUBSTRUCTURE, ETC...]

(the only guaranteed-planar element is a triangle. get into why quads and higher
elements are their own form of parametric surface and they should
not be considered here because of their nebulous definition)
