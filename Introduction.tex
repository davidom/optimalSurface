This work explores optimal mesh representation of surfaces using 
triangular surface meshes.
Most mesh generation algorithms refine an initial discretization based
on {\it a priori} error estimates or quality metrics. For example, the
advancing-front algorithm advances boundaries into space to generate a
grid \cite{tristrano98,diaz-morcillo98}. Other methods generate
grids from iterative refinement or enrichment from initial, coarse
configurations \cite{marcum98,marcum00,shewchuk02}. These mesh
generation strategies focus on element quality---with the
justification being that downstream applications require high quality
geometries in order to achieve a desired level of accuracy. These
approaches, both historically and in a modern sense, have been 
spectacularly successful at producing high quality meshes for 
computational simulation.

Traditionally, surface grid generation processes produce good quality
grids from the combination of geometric growth rates and smoothing.
However, the process requires input, and if the input, or starting place,
is not appropriate, then the geometry is often under and/or oversampled
for the intended use. That fact is not an indictment of the grid
generation process, but instead implies that the final grid is heavily
dependent on the inputs. In addition, if some way of controlling the
point spacing in the middle of the surface (away from the boundaries) is
not present, then more points could be wasted/omitted in an attempt to
accurately represent the geometry. Efforts have gone into
creating a locally or globally optimal surface grid. For example,
curvature is a common driver of surface mesh adaptation or refinement
\cite{siqueria13}. Interpolation error, given a desired function, 
is also ubiquitous \cite{peraire87,alauzet06,buscaglia97,huang05}.

The work presented here is contrasted with the above work in that only
the degree in which the discretization approximates the geometry is
considered.  Here formal definitions of fundamental mesh operations
(triangle split, edge split, edge collapse, and vertex movement) are
developed with respect to a locally optimal representation via 
triangular surface meshes for a continuous, parameterized surface. The 
proposed algorithm is a general-use method
that can be applied to any three-dimensional surface regardless of its 
representation.
This is due to every step being developed without the use of
derivatives. A result of not using derivatives is that each step in the
algorithm is robust to large or discontinuous changes in derivatives.

This process can only be automated if some way of judging ``how well'' a
surface grid represents a surface is present. To this end, a method of
generating surface grids through constrained, local optimization is
detailed below. Results for mesh representation and efficiency are also
given and discussed.  Further discussion of robustness and mesh quality
is also presented.

Element quality is not considered here as a constraint or goal of the 
optimization procedure, as there
are numerous methods for surface mesh quality optimization
\cite{frey1998,garimella2002,garimella2004a,garimella2004b,jiao2005,jiao2006,jiao2011,jiao2013,lopez2008,montenegro2005,montenegro2006,montenegro2008,montenegro2011,roca2012,roca2013,shivanna2010,zhang2009}
which could be used in a postprocessing step if greater mesh quality is
required.  Instead only optimal representation of the underlying
geometry via triangular surface meshes is studied.
