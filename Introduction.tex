Computational design and analysis has become a fundamental part of industry and
academia for use in research, development, and manufacturing. In general, the
accuracy of computational analysis depends heavily on the fidelity of the
computational representation of a real-world object or phenomenon. However, the
task of creating high fidelity models of an actual geometry can be
time-consuming--sometimes consuming up to seventy-five percent of the time
required to produce a solution [REFERENCE]. This work seeks to explore the
concept of optimal surface grid refinement through the development of a formal
description Surface grid generation is a rich and mature topic.  In order to
focus only on the contribution presented here, only a limited background or
explanation will be given for each related topic.  To accomplish this focused
presentation, the author will assume the reader possesses a working knowledge
of commonly used phrases and concepts associated with discrete geometry and
mesh generation.  of fundamental mesh operations. These mesh operations
(triangle split, edge split, edge collapse, edge swap, and vertex smoothing)
were developed with respect to optimal representation of a discrete element
for a continuous, parameterized surface. The concept of discrete element
quality is also defined with respect to optimal representation. The above
concepts are contrasted with traditional methods for grid generation through
direct comparison of representation optimality and efficiency.
[REWORD]
