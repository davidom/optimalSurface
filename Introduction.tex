Computational design and analysis has become a fundamental part of
industry and academia for use in research, development, and
manufacturing. In general, the accuracy of computational analysis
depends heavily on the fidelity of the computational representation of a
real-world object or phenomenon. However, the task of creating high
fidelity models of an actual geometry can be time-consuming---sometimes
consuming up to seventy-five percent of the time required to produce a
solution \cite{bischoff05}. Most mesh generation strategies focus on
element quality---with the justification being that downstream
applications require high quality geometries in order to achieve a
desired level of accuracy. However, element quality should be secondary
to accurately representing the underlying physical object or phenomenon.

The justification for the development of these methods was to explore
the concept of optimal mesh refinement, and subsequently optimal
representation, through the development of a formal description of
fundamental mesh operations. These mesh operations (triangle split, edge
 split, edge collapse, edge swap, and vertex smoothing) were
developed with respect to optimal representation of a discrete element
for a continuous, parameterized surface. The proposed algorithm is a
general-use method that can be applied to any ``digital surface''
regardless of its representation. This is due to every step being
developed without the use of derivatives. Most other methods operate on
a specific type of surface geometry, e.g. NURBS surfaces, and use the
specific information available for the type of surface in use.
[TRANSITION SENTENCE] A result of not using derivatives is that each
step in the algorithm is robust to large changes in derivatives or
curves that are not ``well behaved'', e.g. the were highly oscillatory.

This process can only be automated if some way of judging ``how well'' a
surface grid represents a surface is present. To this end, a method of
generating surface grids through constrained, local optimization is
detailed below. The concept of discrete element quality is also defined
with respect to optimal representation. the above concepts are
contrasted with traditional methods for grid generation through direct
comparison of representation optimality and efficiency.



%NURBS surfaces are one of, if not {\it the}, de-facto
%standard in CAD; however, in other fields, such as pattern recognition,
%other types of digital surfaces, such as parametric, are more common
%[CITE]. T-splines are also becoming more popular in isogeometric
%analysis, for example, [CITE].
%Therefore, a general algorithm that
%does not require derivative information to generate a suitable surface
%grid has been developed. 
