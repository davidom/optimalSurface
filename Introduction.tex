Most mesh generation algorithms refine an initial discretization based on
a priori error estimates [EXAMPLE REFERENCES]. This approach,
historically and in a modern sense, has been spectacularly successful in
their efforts to produce high quality meshes for computation simulation.
Aforementioned mesh generation strategies focus on element quality ---
with the justification being that downstream applications require high
quality geometries in order to achieve a desired level of accuracy.
However, element quality should be secondary to accurately representing
the underlying physical object or phenomenon.  This work seeks to
explore the concept of optimal mesh representation instead of optimal
mesh quality.

Here a formal definition of fundamental mesh operations (triangle split,
edge split, edge collapse, and vertex smoothing) is developed with
respect to optimal representation for a continuous, parameterized
surface. The proposed algorithm is a general-use method that can be
applied to any surface regardless of its representation. This is due to
every step being developed without the use of derivatives. A result of
not using derivatives is that each step in the algorithm is robust to
large or discontinuous changes in derivatives.

This process can only be automated if some way of judging ``how well'' a
surface grid represents a surface is present. To this end, a method of
generating surface grids through constrained, local optimization is
detailed below. The above concepts are contrasted with traditional
methods for grid generation through direct comparison of representation
optimality and efficiency. Further discussion of robustness and mesh
quality is also presented.

In general, grid generation is a name for any process that creates a
grid. For example, the advancing-front algorithm advances boundaries
into space to generate a grid \cite{tristrano98, diaz-morcillo98}[CITE].
Other methods generate grids from iterative refinement or enrichment
from initial, coarse configurations \cite{marcum98, marcum00,
shewchuk02}[CITE]. Usually the benchmark for separating the two
methods, generation and refinement, are the prioritization of grid
quality and grid accuracy (both of these issues will be addressed
later).

Traditionally, surface grid generation processes produce good quality
grids from the combination of geometric growth rates and smoothing.
However, the process requires input and if the input, or starting place,
is not appropriate, then the geometry is often under and/or over sampled
for the intended use. That fact is not an indictment of the grid
generation process, but instead implies that the final grid is heavily
dependent on the inputs. In addition, if some way of controlling the
point spacing in the middle of the surface (away from the boundaries) is
not present, then more points could be wasted/omitted in an attempt to
accurately represent the geometry.

Other efforts have gone into creating a locally or globally optimal
surface grid. Many names have been assigned to the particular task, but
the underlying goal is very similar --- represent a surface as
accurately as possible. For example, [CITE, EXPLAIN, CITE, EXPLAIN,
CITE, EXPLAIN]
